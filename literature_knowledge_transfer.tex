Scholtz and Wiedenbeck {\cite scholtz_learning_1990} conducted a study to identify and characterize the kinds of knowledge that transfer from a known programming language (Pascal or C) to a new, unknown language (Icon). They studied the protocols of four student programmers, which included video captures (capturing the programmers' verbalizations and the computer screen) and their final code output. The students in the study were knowledgeable in either Pascal or C. In analyzing the video data, episodes were classified into whether they involved syntactic, semantic, or planning level of programming knowledge; planning knowledge were broken down into three levels: strategic, tactical, and implementation plans. Their findings suggest that the transfer of syntactic and semantic knowledge were relatively minor problems for the programmers; while knowledge of a previous language influenced how they learned syntax, the errors they encountered were more slips rather than misconceptions. The problems encountered in the transfer of semantic knowledge involved cases when the semantics of a similar construct differed from what they have learned in a previous language (i.e. contructs with the same name in both languages but different semantics). 

The researchers found that the main source of difficulties lies in the programmers' planning knowledge, specifically in the interplay of tactical and implementation plans. Language-independent strategic plans describe a programmer's global strategy for solving a problem. It was found that familiarity with the nature of and solutions to a problem, even in a different language, reduced the need for strategic planning; students simply applied well-learned repertoires of strategies that worked on similar problems. Also language-independent, tactical plans were more specific than strategic plans, focusing on local strategies or algorithms for solving problems. Implementation plans dealt with the specifics of how to achieve the language-independent strategic and tactical plans in the programming language at hand. 

Their findings were that students devise language-agnostic tactical plans from previous experience in a known language and implement them; if familiar constructs don't exist in the new language, there is a failure to implement the plan, which leads to a revision of the tactical plan. The students were likely to enter a cycle of revising the tactical plan, followed by implementation, until a solution finally works. Programmers use well-understood tactical plans when learning new languages and these tactical plans may influence their learning. These results were in line with ideas in the common elements theory of transfer, which states that knowledge will transfer to another only if there are shared common elements between previous knowledge and to-be-learned information {\cite polson_a_quantitative_1987, polson_test_1986, polson_transfer_1987}. Findings in a study by Wu and Anderson {\cite wu_problem-solving_1990} exploring knowledge transfer among programming languages similarly echoed the idea of commonalities in knowledge representation as constituting the basis for transfer. Their work explored transfer of knowledge between the programming languages Lisp and Prolog and between Lisp and Pascal. Based on the results of their study, they proposed three levels of transfer between programming languages: syntactic, algorithmic, and problem levels. This idea is comparable to Scholtz and Weidenbeck's conclusions on the interplay of tactical and implementation plans as affecting knowledge transfer.